\problem{设$g:\mathbb{N}\to\mathbb{N}\in\PRF,f:\mathbb{N}^2\to\mathbb{N}$, 满足
$$\protect\begin{aligned}
    f(x,0)&=g(x),\\
    f(x,y+1)&=f(f(\cdots f(f(x,y),y-1),\cdots),0),
   \protect\end{aligned}
$$证明:$f\in\PRF$.}
\begin{proof}
    先证明$f(x,y)=g^{2^{y-1}}(x)$当$y\geqslant 1$时成立.

    当$y=1$时, $f(x,y)=f(x,0)=g(x)$, 成立;

    假设当$y\leqslant k$时命题成立, 那么有$f(x,y)=g^{2^{y-1}}(x)$在$1\leqslant y\leqslant k$时对任意的$x\in\mathbb{N}$均成立.

    当$y=k+1$时, $$\begin{aligned}
        f(x,k+1)&=f(f(\cdots f(f(x,k),k-1),\cdots),0)\\
        &=f(f(\cdots f(g^{2^{k-1}}(x),k-1),\cdots), 0)\\
        &=f(f(\cdots g^{2^{k-2}}(g^{2^{k-1}}(x)),\cdots), 0)\\
        &=f(g^{2^0}\circ g^{2^1}\circ \cdots \circ g^{2^{k-2}}\circ g^{2^{k-1}}(x),0)\\
        &=g(g^{\frac{1-2^k}{1-2}}(x))\\
        &=g^{2^k}(x)
    \end{aligned}$$
	
    由数学归纳法可知, $f(x,y)=g^{2^{y-1}}(x)$.
    
    现在只要证明$g^{2^{y-1}}(x)\in\PRF$.

    由于$g(x)\in\PRF$, 所以$G(x,n)=g^n(x)\in\PRF$(由定义1.27可知$G(x,n)\in\mathcal{ITF}$, 而$\mathcal{ITF}=\PRF$). 所以只要证明$$F(y)=\begin{cases}
        2^{y-1},&y>0\\
        1,&y=0,
    \end{cases}$$
	
	$F(y)\in\PRF$即可.
    而$$F(y)=\left\lfloor\frac{2^yN^2(y)+2N(y)}{2}\right\rfloor,$$
	
	所以$F(y)\in\PRF$.
    综上, $f\in\PRF$.
\end{proof}