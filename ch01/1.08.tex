\problem{设$$M(x)=\protect\begin{cases}
    M(M(x+11)), & \textrm{若}x\leqslant 100,\\
    x-10, & \textrm{若}x>100,
\protect\end{cases}$$证明:
$$M(x)=
\protect\begin{cases}
    91, & \textrm{若}x\leqslant 100,\\
    x-10, & \textrm{否则}.
\protect\end{cases}$$}
\begin{proof}
    显然, 我们只要证明当$0\leqslant x\leqslant 100$时, $M(x)=91$即可.

    当$90\leqslant x \leqslant 100$时, $M(x)=M(M(x+11))=M(x+1)$, 因此$M(90)=M(91)=\cdots=M(100)=M(101)=91$, 注意$M(91)=91$;

    当$0\leqslant x \leqslant 100$时, 存在自然数$k$使得$90\leqslant x+11k\leqslant 100$成立. 因此$M(x)=M^2(x+11\cdot1)=M^{k+1}(x+11k)=M^kM(x+11k)=M^k(91)=91$.
\end{proof}